% Bakalarska praca
% Autor: Peter Koprda (xkoprd00@stud.fit.vutbr.cz) 
% 2021/2022


% \chapter{Úvod}

% \todo{TODO: Napísať úvod}

% \blindtext[3]

\chapter{Mapové dáta}
\label{map-data}

\section{OpenStreetMap}
OpenStreetMap~\cite{openstreet} je projekt, ktorý vznikol za účelom vytvárania geografickej databázy celého sveta. Cieľom tohto projektu je mať časom záznam o každom geografickom prvku na~planéte. Zatiaľ čo to začalo mapovaním ulíc, postupom času tento projekt zahŕňa chodníky, budovy, vodné cesty, potrubia, lesy, pláže, poštové schránky a dokonca aj jednotlivé stromy.

\subsection{História projektu}
Projekt OpenStreetMap má svoj začiatok v auguste v roku 2004, kedy britský programátor Steve Coast experimentoval s USB GPS prijímačom. Použil softvér nazývaný GPSDrive, ktorý bral mapy z Microsoft MapPoint, ale porušoval licenčné podmienky. Coast vedome nechcel porušovať autorské práva týchto máp, preto hľadal alternatívu, ktorá by neporušovala licenčné podmienky, tú ale nenašiel. Zistil, že neexistujú zdroje mapových dát, ktoré by mohol používať v otvorenom softvére bez toho, aby porušoval licenčné podmienky alebo platil obrovské sumy peňazí. Po odprezentovaní jeho nápadu o vytvorení vlastnej mapy na konferencii otvorených softvérov v Londýne zistil, že viacerí ľudia mali podobný nápad alebo ich Coastov nápad zaujal, a tak vznikla skupina OpenStreetMap.

V začiatkoch bol dátový model príliš simplistický, pretože obsahoval iba jednoduché čiary nakreslené cez informácie Landsat od NASA. V marci v roku 2006 bola vytvorená prvá editovacia aplikácia pre OpenStreetMap \--- JOSM\footnote{\url{https://josm.openstreetmap.de/}}. Po chvíli bola v tomto roku vytvorená prvá plnofarebná mapa mesta Weybridge. V máji toho roku sa usporiadala prvá spoločná akcia, na ktorej bolo úlohou zmapovať ostrov Wight. Bolo to prvýkrát, kedy sa stretlo viacero mapovačov a znamenalo to pre nich prelomový bod projektu, pretože bola vytvorená detailná mapa. Takéto akcie OpenStreetMap komunity sa začali konať častejšie a boli usporadúvané po celom svete.

V auguste v roku 2006 bola vytvorená nadácia \emph{OpenStreetMap Foundation}, ktorou úlohu je podporovať, ale nie kontrolovať OpenStreetMap projekt. Venuje sa podpore rastu, rozvoja a distribúcii voľne dostupných geografických dát a poskytovaniu geografických údajov komukoľvek na používanie a zdieľanie.

Serverový softvér bol pôvodne napísaný v programovacom jazyku Java, ale v~máji v~roku 2007 bola implementácia softvéru prepísaná do platformy Ruby on Rails\footnote{\url{https://rubyonrails.org/}}. Časom ako začal projekt postupne narastať, začali svojimi dátami prispievať súkromné spoločnosti, mestá, ale aj štáty~\cite{bennett2010openstreetmap}. Vo februári v roku 2008 bolo zaregistrovaných \numprint{25000} užívateľov na~stránke OpenStreetMap, v marci v roku 2009 to už bolo \numprint{100000} užívateľov. Počet užívateľov stále rastie, v čase písanie tohto textu je to už viac ako 8,3 milióna zaregistrovaných užívateľov.

\subsection{Dátový model OpenStreetMap}
Prvky (anglicky elements) sú základnou stavebnou súčasťou dátového modelu OpenStreetMap slúžiace k popisu reálneho sveta. Medzi základné prvky OpenStreetMap projektu patria uzol, cesta a relácia. Tieto prvky je možné popísať značkami.

\textbf{Uzol\footnote{\url{https://wiki.openstreetmap.org/wiki/Node}}} (anglicky node) označuje konkrétny bod na povrchu Zeme, je určený svojou zemepisnou šírkou a dĺžkou. Skladá sa minimálne z dvojice súradníc a svojho jednoznačného identifikačného čísla (id).

\textbf{Cesta\footnote{\url{https://wiki.openstreetmap.org/wiki/Way}}} (anglicky way) je usporiadaný zoznam 2 až \numprint{2000} uzlov, ktoré definujú lomenú čiaru. Cesta má aspoň jednu značku alebo je vložená do relácie.

\textbf{Relácia\footnote{\url{https://wiki.openstreetmap.org/wiki/Relation}}} (anglicky relation) sa skladá z jednej alebo viacerých značiek a usporiadaného zoznamu jedného alebo viacerých uzlov alebo ciest. Každý prvok relácie je tzv. člen (anglicky member). Používa sa k popisu závislosti medzi rôznymi prvkami. Každý člen relácie môže voliteľne mať nejakú rolu, ktorá popisuje jeho význam v rámci relácie.

\textbf{Značka\footnote{\url{https://wiki.openstreetmap.org/wiki/Tags}}} (anglicky tag) sa skladá z kľúča a hodnoty. Každá značka popisuje určitú vlastnosť dátových prvkov (uzlov, ciest a relácií) alebo sadu zmien. Kľúč popisuje tému, kategóriu alebo typ mapového prvku (napr. cesta \--- \emph{highway} alebo názvy \--- \emph{name}). Hodnota konkretizuje vlastnosť, ktorú všeobecne popisuje kľúč. Napríklad značka \texttt{highway=residential} predstavuje cestu, ktorá vedie obytnou oblasťou.


\section{GIS}
Geografický informačný systém (GIS) je informačný systém, ktorý sa využíva na získavanie, analyzovanie, vizualizáciu a manažment dát s~priestorovým alebo mapovým vyjadrením. GIS spracováva geografické údaje v~digitálnej podobe. Časť takýchto údajov vzniká napr. pomocou satelitných údajov, meraním pomocou polohového systému GPS alebo inými meracími prístrojmi. Údaje v~papierovej podobe je nutné digitalizovať. Súčasťou GIS je hardvér (počítače, servery, zariadenia na zber dát,...), softvér (špecializované programy pre prácu s~priestorovými dátami), dáta (priestorové údaje) a~používatelia (spracovatelia dát, administrátori GIS a~prijímatelia priestorových informácií)~\cite{introductiontogis}. Používatelia systému môžu využívať rôzne metódy spracovania geografických údajov, ktoré umožňujú údaje prehľadávať, triediť, reklasifikovať, transformovať a~modelovať~\cite{hofierka2003gis}.

Geografické dáta v GIS môžu byť organizované dvomi základnými modelmi \--- vektorovým a rastrovým modelom~\cite{holman2014priestorovedata}.

\subsection{Vektorový model}
Vektorový model je nazývaný podľa spôsobu vyjadrenia jeho jednotlivých častí \--- úseky kriviek s definovanou veľkosťou a smerom \--- \emph{vektorom}. Vektorový model pracuje s~tromi variantami objektov:
\begin{itemize}
    \item \textbf{Bod} (point) \--- reprezentuje jednotlivý pár súradníc X, Y. Takýto objekt má príliš malú plochu na to, aby sa dal zobraziť ako plocha (napr. vodopád, vysielač).
    \item \textbf{Línia} (line) \--- reprezentuje množinu usporiadaných bodov \--- uzlov, ktoré sú pospájané do línie. U~takýchto geometrických útvarov je možné vypočítať dĺžku, ale nie je možné zistiť obsah plochy (napr. cesta, rieka).
    \item \textbf{Polygón} (polygon) \--- reprezentuje množinu usporiadaných bodov \--- uzlov, ktoré sú navzájom pospájané líniami do uzatvorenej plochy. U~týchto geometrických útvarov je možné vypočítať obvod a~veľkosť plochy (napr. vodná plocha, lúka).
\end{itemize}

%\begin{figure}[h]
%    \centering
%    \includegraphics{obrazky-figures/placeholder.pdf}
%    \caption{Vektorový model}
%    \label{fig:vectordata}
%\end{figure}

Existujú rôzne modely dát pomocou ktorých je možné reprezentovať geografické objekty s~využitím vektorovej grafiky:
\begin{itemize}
    \item \textbf{Špagetový model} \--- vychádza z postupov využívaných pri digitalizácií máp. Každý objekt na mape je reprezentovaný jedným záznamom a je uložený ako reťazec X, Y súradníc. Z tohto modelu nie je možné získať žiadne informácie o vzťahoch medzi jednotlivými subjektami aj keď sa jedná o priestorové dáta.
    \item \textbf{Topologický model} \--- každá línia tohto modelu začína a končí v uzlu. Všetky informácie tohto modelu sú ukladané do tzv. \emph{topologických tabuliek} \--- tabuľka spojov, súradníc a polygónov. Model vďaka tomu uchováva priestorové vzťahy medzi objektami.
    \item \textbf{Hierarchický model} \--- ukladá zvlášť informáciu o bodoch, líniách a plochách v~hierarchickej štruktúre pre jednoduchšie vyhľadávanie v dátach. V modeli sú takisto zahrnuté aj odkazy medzi jednotlivými druhmi objektov a obsahuje topologickú informáciu. Tento model je pre manipuláciu a vyhľadávanie v dátach najvhodnejší.
\end{itemize}

\subsubsection{Vektorové formáty dát}
Medzi najznámejšie vektorové formáty dát patrí formát \emph{shapefile} vytvorený spoločnosťou ESRI. Tento formát sa ukladá do zariadenia ako skupina aspoň troch súborov, ktoré síce majú rovnaký názov, ale majú rôznu príponu:
\begin{itemize}
    \item hlavný súbor \texttt{*.shp} \--- obsahuje popis geometrie každého záznamu
    \item indexový súbor \texttt{*.shx} \--- prepája prvok v hlavnom súbore so záznamom v atribútovej tabuľke
    \item databázový súbor \texttt{*.dbf} \--- databázový súbor, obsahuje dáta atribútov pre každý záznam
\end{itemize}

Okrem hore zmienených súboroch môže mať \emph{shapefile} aj iné súbory, ktoré obsahujú ďalšie informácie o dátach:
\begin{itemize}
    \item projekčný súbor \texttt{*.prj} \--- ukladá informácie o súradnicovom systéme
    \item priestorové indexy \texttt{*.gix}, \texttt{*.sbn}, \texttt{*.sbx} \--- umožňujú rýchlejšie vyhľadávanie prvkov
    \item atribútové indexy \texttt{*.atx} \--- urýchľujú vyhľadávanie v atribútovej tabuľke
    \item metadátový súbor \texttt{*.shp.xml} \--- metadáta o zvolenom prvku
    \item kódovací súbor \texttt{*.cpg} \--- súbor pre správnu identifikáciu znakov
\end{itemize}

Existuje veľa iných vektorových formátov dát, ktoré sa používajú v GIS ako napr. \emph{KML}, \emph{KMZ}, \emph{GPX} a \emph{GeoJSON}

\textbf{KML} (Keyhole Markup Language) je dátový formát vyvinutý pre aplikáciu Google Earth. Okrem ukladania informácii o geometrii obsahuje možnosti konfigurácie pre Google Earth mapy.

\textbf{KMZ} (Keyhole Markup Zipped) je dátový formát, ktorý sa takisto používa v aplikáciách Google. Tento formát je rozšírením KML formátu, pretože obsahuje okrem textového popisu aj obrázky tvoriace 3D vizualizácie prvku.

\textbf{GPX} (GPS Exchange Format) je formát údajov GPS pre ukladanie bodov, trás a ich atribútov. Ukladá informácie pomocou textu, podobne ako súbory typu KML.

\textbf{GeoJSON}~\cite{rfc7946} je formát založený na formáte JSON (Javascript Object Notation\footnote{\url{https://www.json.org}}). Tento formát je navrhnutý pre reprezentáciu jednoduchých priestorových geografických dát a ich atribútov. Vo výpise~\ref{lst:geojson} je možné vidieť, ako je reprezentovaný bod vo formáte GeoJSON.

\lstset{
    caption={\texttt{FeatureCollection} obsahuje vo vlastnosti \texttt{features} objekt typu \texttt{Feature}, ktorý v \texttt{properties} obsahuje informácie viazané na objekt \texttt{geometry}.},
    label={lst:geojson},
    basicstyle=\ttfamily\footnotesize\bfseries,
    xleftmargin=.2\textwidth, xrightmargin=.2\textwidth
}
\begin{lstlisting}
{
  "type": "FeatureCollection",
  "features": [
    {
      "type": "Feature",
      "properties": {
        "kraj": "Moravsko-sliezsky",
      },
      "geometry": {
        "type": "Point",
        "coordinates": [
          17.60100156068802,
          49.196293610455584
        ]
      }
    }
  ]
}
\end{lstlisting}

\subsection{Rastrový model}
Rastrová reprezentácia mapových dát sa na rozdiel od vektorovej zameriava na zemský povrch. Používa sa skôr na javy ako je napr. úhrn zrážok či nadmorská výška.

Základným princípom tohto formátu je pokrytie zemského povrchu pravidelnou alebo nepravidelnou sieťou, pričom jednotka tieto siete je \emph{bunka} (pixel, cell). Nepravidelná sieť má výhodu oproti nepravidelnej siete takú, že pomocou nepravidelnej siete je možné jednoduchšie reprezentovať rôzne prechody z roviny na terénnu hranu. Na druhú stranu použitie nepravidelnej siete je výpočtovo a aj algoritmicky náročnejšie. Keďže v tomto modeli neexistujú objekty známe z vektorového modelu GIS, bunky definujú vlastnú hodnotu sledovaného javu v konkrétnej časti priestoru. Hodnota v jednej bunke odpovedá bodu, rada spojených buniek s rovnakou hodnotou odpovedá línii a skupina navzájom susediacich buniek odpovedá ploche.

\begin{figure}[h]
    \centering
    \includegraphics[width=0.5\linewidth]{obrazky-figures/vector-and-raster-data.png}
    \caption{\textbf{Vektorový a rastrový dátový model}\protect\footnotemark.}
    \label{fig:vectorandraster}
\end{figure}

\footnotetext{Prevzaté z~\cite{jukil2017mapdata}}

\subsection{ArcGIS}
V súčasnosti existujú rôznorodé softvéry pre GIS. Medzi najznámejšie patrí ArcGIS od~spoločnosti ESRI, ktorý slúži na mapovanie a priestorovú analýzu navrhnutý tak, aby podporoval poslanie a obchodné ciele organizácií. Tento systém poskytuje tri úrovne licencií. Typ zvolenej licencie rozhoduje o tom, ako sú uložené dáta a ako je možné ich editovať. 

Medzi najznámejšie aplikácie systému ArcGIS patrí ArcMap (použiteľná na priestorové analýzy, editáciu dát a tvorbu kartografických výstupov), ArcCatalog (pomáha organizovať a spravovať všetky dáta), ArcGIS Explorer (volne dostupný prehliadač priestorových dát), ArcGIS for~Server (serverové riešenie pre GIS, umožňuje jednoduchú konfiguráciu webových aplikácií a poskytuje kompletné vývojárske prostredie pre \emph{.NET} a \emph{Java}).

\subsection{ArcČR 500}
ArcČR 500~\cite{arcgis} je digitálna vektorová geografická databáza Českej republiky, spracovaná na úrovni podrobnosti 1 : \numprint{500000}. Obsahom databázy sú prehľadné geografické informácie o~ČR. Zdrojom dát pre geografické dáta ArcČR 500 v 3.3 je databáza Data200, čo je národná vektorová geografická databáza Zeměměřického úřadu (ZÚ) odpovedajúca presnosťou a~stupňom generalizácie 1 : \numprint{200000}. Vstupné dáta z Data200 majú deklarovanú absolútnu presnosť do 100 m. Absolútna polohová odchýlka ArcČR 500 v 3.3 je odhadovaná do 200~m.

Dáta sú uchovávané iba v GIS formátoch firmy ESRI a to vo formáte súborovej databázy. ArcČR 500 je zložená z dvoch geodatabází \--- geografické prvky a administratívne členenie.

\subsubsection*{Geografické prvky}
Geografické prvky boli odvodené zo 17-tich vrstiev databázy Data200. Vrstvy súborovej databázy \texttt{ArcCR500\_v33.gdb} ako aj ich popis a typ prvkov je možné vidieť v tabuľke~\ref{tab:arccr500}.

\begin{table}[H]
\begin{tabular}{|l|l|l|}
    \hline
    \textbf{vrstva}        & \textbf{popis}                                 & \textbf{typ prvku} \\ \hline
    Letiste                & Letisko                                        & bod                \\
    SidlaBody              & Sídla nad 500 obyvateľov                       & bod                \\
    VyskoveKoty            & Výškové kóty (vrcholy kopcov)                  & bod                \\
    ZeleznicniStanice      & Železničná stanica                             & bod                \\
    Hranice                & Štátna, krajská a okresná hranica              & línia              \\
    Silnice                & Cesta                                          & línia              \\
    VodniToky              & Vodné toky                                     & línia              \\
    Vrstevnice             & Vrstevnice po 25 m                             & línia              \\
    Zeleznice              & Železnice                                      & línia              \\
    BazinyARaseliniste     & Močiar a rašelinisko väčšie ako 30 ha          & polygón            \\
    Lesy                   & Lesné plochy väčšie ako 30 ha                  & polygón            \\
    SidlaPlochy            & Sídla nad \numprint{5000} obyvateľov           & polygón            \\
    VodniPlochy            & Vodné plochy väčšie ako 15 ha                  & polygón            \\
    ChranenaUzemi          & Národné parky a chránené krajinné oblasti      & polygón            \\
    KladyZakladnichMap     & Klady základných máp ČR                        & polygón            \\
    KladyTopografickychMap & Klady vojenských topografických máp            & polygón            \\
    SouradnicovaSitJTSK    & Súradnicová sieť systému JTSK v intervale 1 km & línia              \\
    ZemepisnaSitETRS89     & Zemepisná sieť v systéme ETR89                 & línia              \\
    ZemepisnaSitWGS84      & Zemepisná sieť v systéme WGS84                 & línia              \\
    DigitalniModelReliefu  & Raster digitálneho modelu reliéfu              & raster             \\
    StinovanyRelief        & Raster tieňovaného modelu reliéfu              & raster             \\ \hline
\end{tabular}
\caption{Súborová databáza ArcCR500\_v33.gdb. Prevzaté z~\cite{arcgis} a upravené.}
\label{tab:arccr500}
\end{table}

Každá vrstva tejto databázy môže nadobúdať rôzne hodnoty atribútov, ktoré sú pre danú vrstvu zadefinované. V tejto práci sú uvedené pre ilustráciu iba vrstvy Letisko (tabuľka~\ref{tab:letisko}) a Hranica (tabuľka~\ref{tab:hranica}). V tabuľke \ref{tab:letisko} je možné vidieť, že atribút \texttt{TYP} môže nadobúdať tri hodnoty \--- t.j.~letisko je buď civilné alebo vojenské, alebo civilné a vojenské. Táto vrstva je zobrazená na mape ako bod. Na druhú stranu vrstva \emph{Hranica}, ktorá sa zobrazuje na mape ako línia, nemá takú variabilitu atribútov ako vrstva \emph{Letisko}. V tabuľke~\ref{tab:hranica} je možné vidieť, že daná vrstva má iba jeden zadefinovaný atribút, ktorý ale môže nadobúdať 3 hodnoty \--- t.j. hranica môže byť štátna, krajská alebo okresná.

\begin{table}[H]
    \centering
    \begin{tabular}{|l|l|l|}
    \hline
    \textbf{meno atribútu}         & \textbf{popis}      & \textbf{nadobúdané hodnoty}   \\ \hline
    \textbf{TYP}          & Typ letiska         & \begin{tabular}[c]{@{}l@{}}1 - civilné\\ 2 - vojenské\\ 3 - civilné a vojenské\end{tabular} \\
    \textbf{NAZEV}        & Meno                & \textit{konkrétne meno}       \\
    \textbf{NAZEV\_ASCII} & Meno (ASCII formát) & \textit{konkrétne meno}       \\
    \textbf{ICAO}         & Kód ICAO            & \textit{konkrétny kód}        \\
    \textbf{STATUT}       & Statut letiska      & \begin{tabular}[c]{@{}l@{}}1 - medzinárodné\\ 2 - vnútroštátne\end{tabular}                 \\ \hline
    \end{tabular}
\caption{Vrstva Letisko (Letiste). Prevzaté z~\cite{arcgis} a upravené.}
\label{tab:letisko}
\end{table}

\begin{table}[H]
    \centering
    \begin{tabular}{|l|l|l|}
        \hline
        \textbf{meno atribútu} & \textbf{popis} & \textbf{nadobúdané hodnoty}  \\ \hline
        \textbf{TYP}  & Typ hranice    & \begin{tabular}[c]{@{}l@{}}1 - štátna\\ 2 - krajská\\ 3 - okresná\end{tabular} \\ \hline
    \end{tabular}
    \caption{Vrstva Hranica (Hranice). Prevzaté z~\cite{arcgis} a upravené.}
    \label{tab:hranica}
\end{table}

\subsubsection*{Administratívne členenie}
Dáta z Českého statistického úřadu (ČSÚ) boli použité pre tvorbu dát administratívneho členenia. Vrstvy súborovej geodatabázy \texttt{AdministrativniCleneni\_v13.gdb} ako aj ich popis sa nachádza v tabuľke~\ref{tab:administrativne-clenenie}.
\begin{table}[H]
    \centering
    \begin{tabular}{|lll|}
    \hline
    \textbf{názov}  & \textbf{popis}                 & \textbf{typ prvku} \\ \hline
    \textbf{ZSJ}    & Základné sídelné jednotky      & bod/polygón        \\
    \textbf{UTJ}    & Územné technické jednotky      & bod/polygón        \\
    \textbf{KU}     & Katastrálne územie             & bod/polygón        \\
    \textbf{MOaMC}  & Mestské obvody a mestské časti & bod/polygón        \\
    \textbf{COB}    & Časti obce                     & bod/polygón        \\
    \textbf{OBCE}   & Obce a vojenské újazdy         & bod/polygón        \\
    \textbf{POU}    & Obce s povereným úradom        & bod/polygón        \\
    \textbf{ORP}    & Obce s rozšírenou pôsobnosťou  & bod/polygón        \\
    \textbf{OKRESY} & Okresy                         & bod/polygón        \\
    \textbf{KRAJE}  & Kraje                          & bod/polygón        \\
    \textbf{STAT}   & Štát                           & bod/polygón        \\ \hline
    \end{tabular}
    \caption{Súborová databáza AdministrativniCleneni\_v13.gdb. Prevzaté z~\cite{arcgis} a upravené.}
    \label{tab:administrativne-clenenie}
\end{table}

\section{Kataster}
Kataster predstavuje register resp. zoznam a používa sa vo viacerých významoch:
\begin{itemize}
    \item súpis nehnuteľného majetku, ktorý bol vytvorený pre daňové a úradné účely
    \item kataster nehnuteľností, katastrálna kniha
    \item katastrálne územie
\end{itemize}

\subsection{Kataster nehnuteľností}
Súčasný kataster nehnuteľností~\cite{baudys-katastranemovitosti} na území Českej republiky mal ekvivalent medzi svojimi právnymi predchodcami \--- tzv.~pozemkovú knihu, ktorá slúžila na majetkoprávne účely. Okrem pozemkovej knihy sa používal aj pozemkový kataster, ktorý slúžil na~daňové účely. Pozemkový kataster sa používal približne do roku 1957 a zapisovanie do~pozemkovej knihy bol ukončený v~roku 1964, pretože nevypovedala o~skutočných a~aktuálnych právnych vzťahoch k~nehnuteľnostiam. Preto bola vytvorená nová pozemková evidencia nehnuteľností, ktorej hlavným účelom bolo zaistiť podklady pre plánovanie národného hospodárstva. Táto evidencia bola vedená až do roku 1992, kedy si spoločenské zmeny vyžiadali založenie dnešného katastru nehnuteľností.

Dnešný kataster nehnuteľností plní úlohu pozemkovej knihy aj pozemkového katastru a slúži aj ako podklad pre geografické informačné systémy. Kataster nehnuteľností je definovaný ako súbor údajov o nehnuteľností v Českej republike. Okrem súpisu a popisu nehnuteľností zahŕňa ich geometrické a polohové určenie pre jednu katastrálnu obec alebo katastrálne územie. Aj keď v českom katastri sú zapísané všetky pozemky, zo stavieb sú zapísané v katastri len tie budovy, ktoré stanovuje katastrálny zákon. Ide o budovy s popisným či evidenčným číslom alebo hlavnú budovu v rámci areálu nehnuteľností. Na obrázku~\ref{fig:katastrmapa} je možné vidieť ako vyzerá katastrálna mapa mestskej časti Brno-Královo Pole. Obrázok je exportovaný zo stránky Českého katastru nemovitostí\footnote{\url{https://nahlizenidokn.cuzk.cz/VyberKatastrMapa.aspx}}, ktorý okrem faktických a právnych informácií o nehnuteľnostiach obsahuje katastrálne mapy a informácie o vlastníkoch nehnuteľností.

\begin{figure}[h]
    \centering
    \includegraphics[width=0.8\textwidth]{obrazky-figures/katastr-ortofo-kralovopole.jpg}
    \caption{Katastrálna mapa s ortofotomapou mestskej časti Brno-Královo Pole. Ružová farba predstavuje budovy, budovy so zelenou farbou sú nehnuteľnosti s~cenovými údajmi k~jednotke (byty alebo nebytové priestory), vyšrafované budovy sú nehnuteľnosti s~cenovými údajmi k~parcele.}
    \label{fig:katastrmapa}
\end{figure}

\subsection{Územný plán}
Vlastníci katastrov nehnuteľností si musia byť istý, že na danom území existuje poriadok, ktorý by mal zaručovať, že ich práva nebudú ohrozované náhodnými a~meniacimi sa rozhodnutiami. Územným plánovaním sa predchádza nekoncepčnému a zložitému rozvoja obce. Zaoberá sa všetkými aspektmi nášho prostredia. Ide prevažne o~stavbu sídiel, dopravnú a~ technickú infraštruktúru, ale aj o prvky, ktoré vytvárajú prírodné zložky životného prostredia~\cite{uzemnyplan}.

% \chapter{Návrh}
% \label{navrh}
% \todo{TODO}
% \blindtext[2]

% \Blindtext


% \chapter{Implementácia}
% \label{implementacia}
% \todo{TODO}

% \blindtext[13]


% \chapter{Testovanie}
% \label{testovanie}
% \todo{TODO}

% \blindtext[1]

% \blindtext[10]



% \chapter{Záver}
% \label{zaver}
% \todo{TODO}

% \blindtext[3]
