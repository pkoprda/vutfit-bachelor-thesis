% Bakalarska praca
% Autor: Peter Koprda (xkoprd00@stud.fit.vutbr.cz) 
% 2021/2022


\chapter{Úvod}

\todo{TODO: Napísať úvod}

\blindtext[3]

\chapter{Mapové dáta} 
\label{map-data}

\section{OpenStreetMap}
OpenStreetMap~\cite{openstreet} je projekt, ktorý vznikol za účelom vytvárania geografickej databázy celého sveta. Cieľom tohto projektu je mať časom záznam o každom geografickom prvku na~planéte. Zatiaľ čo to začalo mapovaním ulíc, postupom času tento projekt zahŕňa chodníky, budovy, vodné cesty, potrubia, lesy, pláže, poštové schránky a dokonca aj jednotlivé stromy.

\subsection{História projektu}
Projekt OpenStreetMap má svoj začiatok v auguste v roku 2004, kedy britský programátor Steve Coast experimentoval s USB GPS prijímačom. Použil softvér nazývaný GPSDrive, ktorý bral mapy z Microsoft MapPoint, ale porušoval licenčné podmienky. Coast vedome nechcel porušovať autorské práva týchto máp, preto hľadal alternatívu, ktorá by neporušovala licenčné podmienky, tú ale nenašiel. Zistil, že neexistujú zdroje mapových dát, ktoré by mohol používať v otvorenom softvére bez toho, aby porušoval licenčné podmienky alebo platil obrovské sumy peňazí. Po odprezentovaní jeho nápadu o vytvorení vlastnej mapy na konferencii otvorených softvérov v Londýne zistil, že viacerí ľudia mali podobný nápad alebo ich Coastov nápad zaujal, a tak vznikla skupina OpenStreetMap.

V začiatkoch bol dátový model príliš simplistický, pretože obsahoval iba jednoduché čiary nakreslené cez informácie Landsat od NASA. V marci v roku 2006 bola vytvorená prvá editovacia aplikácia pre OpenStreetMap \--- JOSM\footnote{\url{https://josm.openstreetmap.de/}}. Po chvíli bola v tomto roku vytvorená prvá plnofarebná mapa mesta Weybridge. V máji toho roku sa usporiadala prvá spoločná akcia, na ktorej bolo úlohou zmapovať ostrov Wight. Bolo to prvýkrát, kedy sa stretlo viacero mapovačov a znamenalo to pre nich prelomový bod projektu, pretože bola vytvorená detailná mapa. Takéto akcie OpenStreetMap komunity sa začali konať častejšie a boli usporadúvané po celom svete.

V auguste v roku 2006 bola vytvorená nadácia \textit{OpenStreetMap Foundation}, ktorou úlohu je podporovať, ale nie kontrolovať OpenStreetMap projekt. Venuje sa podpore rastu, rozvoja a distribúcii voľne dostupných geografických dát a poskytovaniu geografických údajov komukoľvek na používanie a zdieľanie.

Serverový softvér bol pôvodne napísaný v programovacom jazyku Java, ale v~máji v~roku 2007 bola implementácia softvéru prepísaná do platformy Ruby on Rails\footnote{\url{https://rubyonrails.org/}}. Časom ako začal projekt postupne narastať, začali svojimi dátami prispievať súkromné spoločnosti, mestá, ale aj štáty~\cite{bennett2010openstreetmap}. Vo februári v roku 2008 bolo zaregistrovaných $25000$ užívateľov na~stránke OpenStreetMap, v marci v roku 2009 to už bolo $100000$ užívateľov. Počet užívateľov stále rastie, v čase písanie tohto textu je to už viac ako 8,3 milióna zaregistrovaných užívateľov.

\subsection{Dátový model OpenStreetMap}
Prvky (anglicky elements) sú základnou stavebnou súčasťou dátového modelu OpenStreetMap slúžiace k popisu reálneho sveta. Medzi základné prvky OpenStreetMap projektu patria uzol, cesta a relácia. Tieto prvky je možné popísať značkami.

\textbf{Uzol\footnote{\url{https://wiki.openstreetmap.org/wiki/Node}}} (anglicky node) označuje konkrétny bod na povrchu Zeme, je určený svojou zemepisnou šírkou a dĺžkou. Skladá sa minimálne z dvojice súradníc a svojho jednoznačného identifikačného čísla (id).

\textbf{Cesta\footnote{\url{https://wiki.openstreetmap.org/wiki/Way}}} (anglicky way) je usporiadaný zoznam 2 až 2000 uzlov, ktoré definujú lomenú čiaru. Cesta má aspoň jednu značku alebo je vložená do relácie.

\textbf{Relácia\footnote{\url{https://wiki.openstreetmap.org/wiki/Relation}}} (anglicky relation) sa skladá z jednej alebo viacerých značiek a usporiadaného zoznamu jedného alebo viacerých uzlov alebo ciest. Každý prvok relácie je tzv. člen (anglicky member). Používa sa k popisu závislosti medzi rôznymi prvkami. Každý člen relácie môže voliteľne mať nejakú rolu, ktorá popisuje jeho význam v rámci relácie.

\textbf{Značka\footnote{\url{https://wiki.openstreetmap.org/wiki/Tags}}} (anglicky tag) sa skladá z kľúča a hodnoty. Každá značka popisuje určitú vlastnosť dátových prvkov (uzlov, ciest a relácií) alebo sadu zmien. Kľúč popisuje tému, kategóriu alebo typ mapového prvku (napr. cesta \--- \textit{highway} alebo názvy \--- \textit{name}). Hodnota konkretizuje vlastnosť, ktorú všeobecne popisuje kľúč. Napríklad značka \texttt{highway=residential} predstavuje cestu, ktorá vedie obytnou oblasťou.


\section{GIS}
Geografický informačný systém (GIS) je informačný systém, ktorý sa využíva na získavanie, analyzovanie, vizualizáciu a manažment dát s priestorovým alebo mapovým vyjadrením. GIS spracováva geografické údaje v digitálnej podobe. Časť takýchto údajov vzniká napr. pomocou satelitných údajov, meraním pomocou polohového systému GPS alebo inými meracími prístrojmi. Údaje v papierovej podobe je nutné digitalizovať. Súčasťou GIS je hardvér (počítače, servery, zariadenia na zber dát,...), softvér (špecializované programy pre prácu s priestorovými dátami), dáta (priestorové údaje) a používatelia (spracovatelia dát, administrátori GIS a prijímatelia priestorových informácií)~\cite{introductiontogis}. Používatelia systému môžu využívať rôzne metódy spracovania geografických údajov, ktoré umožňujú údaje prehľadávať, triediť, reklasifikovať, transformovať a modelovať~\cite{hofierka2003gis}.

\subsection{ArcGIS}
V súčasnosti existujú rôznorodé softvéry pre GIS. ArcGIS~\cite{arcgis} je systém GIS od spoločnosti Esri, ktorý umožňuje vytvárať interaktívne mapy. Je dostupný v troch úrovniach licencií. 


\chapter{Návrh}
\label{navrh}
\todo{TODO}
\blindtext[2]

\Blindtext


\chapter{Implementácia}
\label{implementacia}
\todo{TODO}

\blindtext[13]


\chapter{Testovanie}
\label{testovanie}
\todo{TODO}

\blindtext[1]

\blindtext[10]



\chapter{Záver}
\label{zaver}
\todo{TODO}

\blindtext[3]
