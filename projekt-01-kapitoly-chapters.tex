% Bakalarska praca
% Autor: Peter Koprda (xkoprd00@stud.fit.vutbr.cz) 
% 2021/2022


% \chapter{Úvod}

% \todo{TODO: Napísať úvod}

% \blindtext[3]

\chapter{Mapové dáta} 
\label{map-data}

\section{OpenStreetMap}
OpenStreetMap~\cite{openstreet} je projekt, ktorý vznikol za účelom vytvárania geografickej databázy celého sveta. Cieľom tohto projektu je mať časom záznam o každom geografickom prvku na~planéte. Zatiaľ čo to začalo mapovaním ulíc, postupom času tento projekt zahŕňa chodníky, budovy, vodné cesty, potrubia, lesy, pláže, poštové schránky a dokonca aj jednotlivé stromy.

\subsection{História projektu}
Projekt OpenStreetMap má svoj začiatok v auguste v roku 2004, kedy britský programátor Steve Coast experimentoval s USB GPS prijímačom. Použil softvér nazývaný GPSDrive, ktorý bral mapy z Microsoft MapPoint, ale porušoval licenčné podmienky. Coast vedome nechcel porušovať autorské práva týchto máp, preto hľadal alternatívu, ktorá by neporušovala licenčné podmienky, tú ale nenašiel. Zistil, že neexistujú zdroje mapových dát, ktoré by mohol používať v otvorenom softvére bez toho, aby porušoval licenčné podmienky alebo platil obrovské sumy peňazí. Po odprezentovaní jeho nápadu o vytvorení vlastnej mapy na konferencii otvorených softvérov v Londýne zistil, že viacerí ľudia mali podobný nápad alebo ich Coastov nápad zaujal, a tak vznikla skupina OpenStreetMap.

V začiatkoch bol dátový model príliš simplistický, pretože obsahoval iba jednoduché čiary nakreslené cez informácie Landsat od NASA. V marci v roku 2006 bola vytvorená prvá editovacia aplikácia pre OpenStreetMap \--- JOSM\footnote{\url{https://josm.openstreetmap.de/}}. Po chvíli bola v tomto roku vytvorená prvá plnofarebná mapa mesta Weybridge. V máji toho roku sa usporiadala prvá spoločná akcia, na ktorej bolo úlohou zmapovať ostrov Wight. Bolo to prvýkrát, kedy sa stretlo viacero mapovačov a znamenalo to pre nich prelomový bod projektu, pretože bola vytvorená detailná mapa. Takéto akcie OpenStreetMap komunity sa začali konať častejšie a boli usporadúvané po celom svete.

V auguste v roku 2006 bola vytvorená nadácia \emph{OpenStreetMap Foundation}, ktorou úlohu je podporovať, ale nie kontrolovať OpenStreetMap projekt. Venuje sa podpore rastu, rozvoja a distribúcii voľne dostupných geografických dát a poskytovaniu geografických údajov komukoľvek na používanie a zdieľanie.

Serverový softvér bol pôvodne napísaný v programovacom jazyku Java, ale v~máji v~roku 2007 bola implementácia softvéru prepísaná do platformy Ruby on Rails\footnote{\url{https://rubyonrails.org/}}. Časom ako začal projekt postupne narastať, začali svojimi dátami prispievať súkromné spoločnosti, mestá, ale aj štáty~\cite{bennett2010openstreetmap}. Vo februári v roku 2008 bolo zaregistrovaných \numprint{25000} užívateľov na~stránke OpenStreetMap, v marci v roku 2009 to už bolo \numprint{100000} užívateľov. Počet užívateľov stále rastie, v čase písanie tohto textu je to už viac ako 8,3 milióna zaregistrovaných užívateľov.

\subsection{Dátový model OpenStreetMap}
Prvky (anglicky elements) sú základnou stavebnou súčasťou dátového modelu OpenStreetMap slúžiace k popisu reálneho sveta. Medzi základné prvky OpenStreetMap projektu patria uzol, cesta a relácia. Tieto prvky je možné popísať značkami.

\textbf{Uzol\footnote{\url{https://wiki.openstreetmap.org/wiki/Node}}} (anglicky node) označuje konkrétny bod na povrchu Zeme, je určený svojou zemepisnou šírkou a dĺžkou. Skladá sa minimálne z dvojice súradníc a svojho jednoznačného identifikačného čísla (id).

\textbf{Cesta\footnote{\url{https://wiki.openstreetmap.org/wiki/Way}}} (anglicky way) je usporiadaný zoznam 2 až \numprint{2000} uzlov, ktoré definujú lomenú čiaru. Cesta má aspoň jednu značku alebo je vložená do relácie.

\textbf{Relácia\footnote{\url{https://wiki.openstreetmap.org/wiki/Relation}}} (anglicky relation) sa skladá z jednej alebo viacerých značiek a usporiadaného zoznamu jedného alebo viacerých uzlov alebo ciest. Každý prvok relácie je tzv. člen (anglicky member). Používa sa k popisu závislosti medzi rôznymi prvkami. Každý člen relácie môže voliteľne mať nejakú rolu, ktorá popisuje jeho význam v rámci relácie.

\textbf{Značka\footnote{\url{https://wiki.openstreetmap.org/wiki/Tags}}} (anglicky tag) sa skladá z kľúča a hodnoty. Každá značka popisuje určitú vlastnosť dátových prvkov (uzlov, ciest a relácií) alebo sadu zmien. Kľúč popisuje tému, kategóriu alebo typ mapového prvku (napr. cesta \--- \emph{highway} alebo názvy \--- \emph{name}). Hodnota konkretizuje vlastnosť, ktorú všeobecne popisuje kľúč. Napríklad značka \texttt{highway=residential} predstavuje cestu, ktorá vedie obytnou oblasťou.


\section{GIS}
Geografický informačný systém (GIS) je informačný systém, ktorý sa využíva na získavanie, analyzovanie, vizualizáciu a manažment dát s~priestorovým alebo mapovým vyjadrením. GIS spracováva geografické údaje v~digitálnej podobe. Časť takýchto údajov vzniká napr. pomocou satelitných údajov, meraním pomocou polohového systému GPS alebo inými meracími prístrojmi. Údaje v~papierovej podobe je nutné digitalizovať. Súčasťou GIS je hardvér (počítače, servery, zariadenia na zber dát,...), softvér (špecializované programy pre prácu s~priestorovými dátami), dáta (priestorové údaje) a~používatelia (spracovatelia dát, administrátori GIS a~prijímatelia priestorových informácií)~\cite{introductiontogis}. Používatelia systému môžu využívať rôzne metódy spracovania geografických údajov, ktoré umožňujú údaje prehľadávať, triediť, reklasifikovať, transformovať a~modelovať~\cite{hofierka2003gis}.

\subsection{ArcGIS}
V súčasnosti existujú rôznorodé softvéry pre GIS. Medzi najznámejšie patrí ArcGIS od~spoločnosti Esri, ktorý slúži na mapovanie a priestorovú analýzu navrhnutý tak, aby podporoval poslanie a obchodné ciele organizácií. Tento systém poskytuje tri úrovne licencií. Typ zvolenej licencie rozhoduje o tom, ako sú uložené dáta a ako je možné ich editovať. 

Medzi najznámejšie aplikácie systému ArcGIS patrí ArcMap (použiteľná na priestorové analýzy, editáciu dát a tvorbu kartografických výstupov), ArcCatalog (pomáha organizovať a spravovať všetky dáta), ArcGIS Explorer (volne dostupný prehliadač priestorových dát), ArcGIS for~Server (serverové riešenie pre GIS, umožňuje jednoduchú konfiguráciu webových aplikácií a poskytuje kompletné vývojárske prostredie pre \emph{.NET} a \emph{Java}).

\subsection{ArcČR 500}
ArcČR 500~\cite{arcgis} je digitálna vektorová geografická databáza Českej republiky, spracovaná na úrovni podrobnosti 1 : \numprint{500000}. Obsahom databázy sú prehľadné geografické informácie o~ČR. Zdrojom dát pre geografické dáta ArcČR 500 v 3.3 je databáza Data200, čo je národná vektorová geografická databáza Zeměměřického úřadu (ZÚ) odpovedajúca presnosťou a~stupňom generalizácie 1 : \numprint{200000}. Vstupné dáta z Data200 majú deklarovanú absolútnu presnosť do 100 m. Absolútna polohová odchýlka ArcČR 500 v 3.3 je odhadovaná do 200~m.

Dáta sú uchovávané iba v GIS formátoch firmy ESRI a to vo formáte súborovej databázy. ArcČR 500 je zložená z dvoch geodatabází \--- geografické prvky a administratívne členenie.

\subsubsection*{Geografické prvky}
Geografické prvky boli odvodené zo 17-tich vrstiev databázy Data200. Vrstvy súborovej databázy \texttt{ArcCR500\_v33.gdb} ako aj ich popis a typ prvkov je možné vidieť v tabuľke~\ref{tab:arccr500}.

\begin{table}[H]
\begin{tabular}{|l|l|l|}
    \hline
    \textbf{vrstva}        & \textbf{popis}                                 & \textbf{typ prvku} \\ \hline
    Letiste                & Letisko                                        & bod                \\
    SidlaBody              & Sídla nad 500 obyvateľov                       & bod                \\
    VyskoveKoty            & Výškové kóty (vrcholy kopcov)                  & bod                \\
    ZeleznicniStanice      & Železničná stanica                             & bod                \\
    Hranice                & Štátna, krajská a okresná hranica              & línia              \\
    Silnice                & Cesta                                          & línia              \\
    VodniToky              & Vodné toky                                     & línia              \\
    Vrstevnice             & Vrstevnice po 25 m                             & línia              \\
    Zeleznice              & Železnice                                      & línia              \\
    BazinyARaseliniste     & Močiar a rašelinisko väčšie ako 30 ha          & polygón            \\
    Lesy                   & Lesné plochy väčšie ako 30 ha                  & polygón            \\
    SidlaPlochy            & Sídla nad \numprint{5000} obyvateľov           & polygón            \\
    VodniPlochy            & Vodné plochy väčšie ako 15 ha                  & polygón            \\
    ChranenaUzemi          & Národné parky a chránené krajinné oblasti      & polygón            \\
    KladyZakladnichMap     & Klady základných máp ČR                        & polygón            \\
    KladyTopografickychMap & Klady vojenských topografických máp            & polygón            \\
    SouradnicovaSitJTSK    & Súradnicová sieť systému JTSK v intervale 1 km & línia              \\
    ZemepisnaSitETRS89     & Zemepisná sieť v systéme ETR89                 & línia              \\
    ZemepisnaSitWGS84      & Zemepisná sieť v systéme WGS84                 & línia              \\
    DigitalniModelReliefu  & Raster digitálneho modelu reliéfu              & raster             \\
    StinovanyRelief        & Raster tieňovaného modelu reliéfu              & raster             \\ \hline
\end{tabular}
\caption{Súborová databáza ArcCR500\_v33.gdb. Prevzaté z~\cite{arcgis} a upravené.}
\label{tab:arccr500}
\end{table}

Každá vrstva tejto databázy môže nadobúdať rôzne hodnoty atribútov, ktoré sú pre danú vrstvu zadefinované. V tejto práci sú uvedené pre ilustráciu iba vrstvy Letisko (tabuľka~\ref{tab:letisko}) a Hranica (tabuľka~\ref{tab:hranica}). V tabuľke \ref{tab:letisko} je možné vidieť, že atribút \texttt{TYP} môže nadobúdať tri hodnoty \--- t.j.~letisko je buď civilné alebo vojenské, alebo civilné a vojenské. Táto vrstva je zobrazená na mape ako bod. Na druhú stranu vrstva \emph{Hranica}, ktorá sa zobrazuje na mape ako línia, nemá takú variabilitu atribútov ako vrstva \emph{Letisko}. V tabuľke~\ref{tab:hranica} je možné vidieť, že daná vrstva má iba jeden zadefinovaný atribút, ktorý ale môže nadobúdať 3 hodnoty \--- t.j. hranica môže byť štátna, krajská alebo okresná.

\begin{table}[H]
    \centering
    \begin{tabular}{|l|l|l|}
    \hline
    \textbf{meno atribútu}         & \textbf{popis}      & \textbf{nadobúdané hodnoty}   \\ \hline
    \textbf{TYP}          & Typ letiska         & \begin{tabular}[c]{@{}l@{}}1 - civilné\\ 2 - vojenské\\ 3 - civilné a vojenské\end{tabular} \\
    \textbf{NAZEV}        & Meno                & \textit{konkrétne meno}       \\
    \textbf{NAZEV\_ASCII} & Meno (ASCII formát) & \textit{konkrétne meno}       \\
    \textbf{ICAO}         & Kód ICAO            & \textit{konkrétny kód}        \\
    \textbf{STATUT}       & Statut letiska      & \begin{tabular}[c]{@{}l@{}}1 - medzinárodné\\ 2 - vnútroštátne\end{tabular}                 \\ \hline
    \end{tabular}
\caption{Vrstva Letisko (Letiste). Prevzaté z~\cite{arcgis} a upravené.}
\label{tab:letisko}
\end{table}

\begin{table}[H]
    \centering
    \begin{tabular}{|l|l|l|}
        \hline
        \textbf{meno atribútu} & \textbf{popis} & \textbf{nadobúdané hodnoty}  \\ \hline
        \textbf{TYP}  & Typ hranice    & \begin{tabular}[c]{@{}l@{}}1 - štátna\\ 2 - krajská\\ 3 - okresná\end{tabular} \\ \hline
    \end{tabular}
    \caption{Vrstva Hranica (Hranice). Prevzaté z~\cite{arcgis} a upravené.}
    \label{tab:hranica}
\end{table}

\subsubsection*{Administratívne členenie}
Vrstvy súborovej geodatabázy \texttt{AdministrativniCleneni\_v13.gdb} ako aj ich popis sa nachádza v tabuľke~\ref{tab:administrativne-clenenie}.
\begin{table}[H]
    \centering
    \begin{tabular}{|lll|}
    \hline
    \textbf{názov}  & \textbf{popis}                 & \textbf{typ prvku} \\ \hline
    \textbf{ZSJ}    & Základné sídelné jednotky      & bod/polygón        \\
    \textbf{UTJ}    & Územné technické jednotky      & bod/polygón        \\
    \textbf{KU}     & Katastrálne územie             & bod/polygón        \\
    \textbf{MOaMC}  & Mestské obvody a mestské časti & bod/polygón        \\
    \textbf{COB}    & Časti obce                     & bod/polygón        \\
    \textbf{OBCE}   & Obce a vojenské újazdy         & bod/polygón        \\
    \textbf{POU}    & Obce s povereným úradom        & bod/polygón        \\
    \textbf{ORP}    & Obce s rozšírenou pôsobnosťou  & bod/polygón        \\
    \textbf{OKRESY} & Okresy                         & bod/polygón        \\
    \textbf{KRAJE}  & Kraje                          & bod/polygón        \\
    \textbf{STAT}   & Štát                           & bod/polygón        \\ \hline
    \end{tabular}
    \caption{Súborová databáza AdministrativniCleneni\_v13.gdb. Prevzaté z~\cite{arcgis} a upravené.}
    \label{tab:administrativne-clenenie}
\end{table}

\chapter{Návrh}
\label{navrh}
\todo{TODO}
\blindtext[2]

\Blindtext


\chapter{Implementácia}
\label{implementacia}
\todo{TODO}

\blindtext[13]


\chapter{Testovanie}
\label{testovanie}
\todo{TODO}

\blindtext[1]

\blindtext[10]



\chapter{Záver}
\label{zaver}
\todo{TODO}

\blindtext[3]
